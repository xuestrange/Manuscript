\documentclass[a4paper, 12pt]{article}


\usepackage{amsmath,amssymb,amsthm,bm} % 数学符号和公式
\usepackage{tikz}  % 绘图包
\usepackage{xcolor} % 用于给超链接指定颜色
\usepackage{float} % 浮动体
\usepackage{graphics} % 插入图片

\usepackage{geometry} % 页边距
\geometry{top=3cm,bottom=3cm,left=2.5cm,right=2.5cm}

\usepackage{fancyhdr} % 页眉页脚设置
\usepackage{lastpage}
\fancypagestyle{plain}{%
    \fancyhf{} % 清空所有预设样式
    \fancyfoot[C]{\thepage{}/\pageref*{LastPage}} % 设置页脚中心位置
    \renewcommand{\headrulewidth}{0.4pt} % 页眉横线的宽度
    \renewcommand{\footrulewidth}{0pt}   % 页脚横线的宽度
}
\pagestyle{plain}

\usepackage{booktabs} % 表格
\usepackage{tabularx}
\usepackage[flushleft]{threeparttable} % 表格下边的脚注, 靠左对齐
\usepackage{caption}  % 图例
\usepackage{setspace} % 空白
% ---------------- Pdf_latex From Inkscape --------------- %
% 用于插入inkscape生成的带有latex标注的图片
\usepackage{import}
\graphicspath{{./figure}} % 声明 pdf_tex和对应pdf所在的文件夹
\usepackage{xifthen}
\usepackage{pdfpages}
\usepackage{transparent}
% -------------------------------------------------------- %
\usepackage{hyperref} % 超链接格式
\usepackage{fixdif} % 积分符号\d
\usepackage[noabbrev]{cleveref} % 交叉引用
% 颜色设置要在cleveref包之后
\def\mycolor{purple}
\hypersetup{colorlinks=true,linkcolor=\mycolor,citecolor=\mycolor,urlcolor=\mycolor}
\usepackage{apacite}
\usepackage{natbib}
\bibliographystyle{apacite}
\begin{document}

\begin{titlepage}
    \title{A Title for This Paper\thanks{abc}}
    % 作者和作者简介
    \author{XJZ\thanks{USTC} \and WLZ\thanks{CPU}}
    \date{\today}
    \maketitle
    \begin{abstract}
        \noindent
        % 摘要
        This is you abstract.
        \\
        \vspace{0in}\\
        % 关键字
        \noindent\textbf{Keywords:} climate, AI, robots\\
        \bigskip
    \end{abstract}
    \setcounter{page}{0}
    \thispagestyle{empty}
\end{titlepage}
\pagebreak \newpage
\doublespacing

\section{Results}\label{sec:results}

\clearpage


\end{document}
